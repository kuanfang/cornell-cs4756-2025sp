\documentclass[11pt]{article}
\usepackage[letterpaper,margin=0.8in]{geometry}
\usepackage{epic}
\usepackage{eepic}
\usepackage{graphicx}
\usepackage{algorithm,algorithmic}
 \usepackage{enumitem}
\usepackage{amsfonts,mathrsfs}
\usepackage{hyperref}
\usepackage{color}
\usepackage{geometry}
\usepackage{changepage}
\usepackage{amsmath}
\pagestyle{empty}
\setlength{\textheight}{9.5in}
\usepackage {tikz}
\usetikzlibrary{shapes,arrows, positioning}
\usepackage{caption}
\DeclareMathOperator{\E}{\mathbb E}
\DeclareMathOperator*{\argmax}{argmax}


\begin{document}
\setlength{\fboxrule}{.5mm}\setlength{\fboxsep}{1.2mm}
\newlength{\boxlength}\setlength{\boxlength}{\textwidth}
\addtolength{\boxlength}{-4mm}
\begin{center}\framebox{\parbox{\boxlength}{\bf
Robot Learning \hfill Assignment 3\\
CS 4756 Spring 2025 \hfill Due 11:59pm Friday,  March 21}}\end{center}
\vspace{5mm}
\setlength{\fboxrule}{0.1pt}\setlength{\fboxsep}{2mm}

\def\ind{\hspace*{0.3in}}
\def\gap{0.2in}

\vskip \gap \noindent{\bf Bounding Error in Approximate Policy Iteration  (10 points)}
\newline

\noindent In this problem, we look at how errors in approximating the value function affect the performance of a policy during policy iteration. We aim to understand how the error $\epsilon$ in the value function propagates and impacts the overall performance of the policy.
\newline

\noindent Let’s assume we are in an infinite horizon MDP with discount factor $\gamma$. We have a reference policy $\pi$ whose true value function is $V^\pi(s)$. We collect rollouts with $\pi$, and fit a neural network to approximate this value function, where $\hat V(s) \approx V^\pi(s)$, given that the value function approximation error is bounded by $\epsilon$.
\newline

\noindent Let’s assume we did a really good job and can guarantee that the error from the fit is at most $\epsilon$. More formally, let $|| V^\pi - \hat V||_{\infty} \leq \epsilon$.\footnote{Note: $|| x ||_{\infty}$ is the L-infinity norm}
\newline

\noindent We now choose a greedy policy to improve upon policy $\hat{\pi}$:
$$\hat \pi(s) = \argmax_a \left[R(s,a) +  \gamma \sum_{s'} P(s'|s,a)\hat V(s')\right]$$
\noindent Note that this is exactly the policy improvement step, except the value function is substituted with our approximate value function. We want to know how the greedy policy  $\hat \pi(s)$ performs with respect to $\pi (s)$. 
\newline

\noindent In other words, $\hat{\pi}$ can end up doing much worse than $\pi$. This shows that even though the error in approximating the value function is at most $\epsilon$, the performance error scales up by a factor of $\frac{1}{1-\gamma}$. 
\newline

\noindent{\bf (a)} \noindent Let $V^{\hat \pi}(s)$ be the value of the greedy policy $\hat \pi(s)$. Prove the following:

$$V^{\pi}(s) -V^{\hat \pi}(s) \leq  \frac{2\gamma \epsilon}{1-\gamma} \text{\;\;, for all $s$}$$

\noindent In other words, $\hat{\pi}$ can end up doing much worse than $\pi$. Additionally, even though the error from fitting the value was $\epsilon$, the performance error scales up by a factor of $\frac{1}{1-\gamma}$. 

\noindent \textit{Hint}:
One way to approach the question would be:
\begin{enumerate}
    \item Start by using the Bellman equation for any policy:
$$V^{\pi}(s) = R(s,\pi(s)) + \gamma \sum_{s'} P(s'|s,\pi(s)) V^{\pi}(s')$$ 
Use this substitution to expand $V^{\pi}(s) -V^{\hat \pi}(s)$.
\item Next, note that you need to establish a relationship between $\pi(s)$ and $\hat \pi(s)$. Exploit the following observation: $\hat \pi(s) = \argmax_a f(s,a) $ must imply $f(s,\hat \pi(s))  \geq f(s,\pi(s))$ for any policy $\pi$.
\item Use these facts to obtain the following intermediate result. For any $s$,
$$V^{\pi}(s) -V^{\hat \pi}(s) \leq 2\gamma \epsilon + \gamma\sum_{s' \in \mathcal S} \Pr[s' | s, \hat \pi(s)](V^\pi(s') - V^{\hat \pi}(s'))$$
From the above, you should be able to prove the final result for all $s$.
\end{enumerate}

\noindent{\bf (b)} \noindent Explain why the performance error bound between $\pi(s)$ and $\hat \pi(s)$ is not simply $\epsilon$. Does the error remain constant, scale with time, or compound over time? Why is the scaling factor $\frac{1}{1-\gamma}$ significant in this bound?
\newline

\noindent \textit{Hint:}
Think about how the value function accumulates rewards over an infinite horizon and how errors in each step affect the future value estimates.
\newline

\noindent{\bf (c)} \noindent What would happen to the performance bound between if the time horizon $T$ were finite instead of infinite? How would the bound change, and what implications does this have for practical applications of approximate policy iteration?

\end{document}
